\documentclass[12pt]{article}
\usepackage{graphicx}
\usepackage{multirow}
\usepackage{fullpage}
\usepackage{times}
\setlength\parindent{0pt}
\setlength\parskip{12pt}

\begin{document}

{\sffamily
\begin{tabular}{ll}
\includegraphics[width=1in]{ach.png}\\
& \textbf{\Huge{SIGBOVIK 2016 Paper Review}} \\ &\\
& \LARGE{Paper 22: Sniffing for meaning:} \\
& \LARGE{Defining and maximizing the signal-to-nose ratio} \\
&\\
\hline
\end{tabular}}
\vspace{2em}

{\large\bf
\begin{tabular}{l}
Harry Ne'austreulls \\
Rating: 4 (strong accept) \\
Confidence: 0/0 \\
\end{tabular}}
\vspace{1em}

As someone who has repeatedly noticed references to the described signal-to-nose ratio but never
been able find an authoritative source on the topic, I was overjoyed to encounter your submission.
It gives the entire PC an inflated ego to think that SIGBOVIK should be the chosen venue for the
canonical paper on any topic, let alone one that practically reeks with demonstrated industry
application.

I would, however, like to critique one of your claims: while reading the second section of your
paper, I smelled a rat.
You claim that the designers of the Sphinx made an attempt at ``doing away with the nose
altogether,'' but then go on to argue that their signal-to-nose ratio is 0.
The astute reader---me for example---will observe that, as you correctly state in the second column
of the same page, any noseless construct will have an undefined ratio.
The presence of this correct statement shows you're someone who nose what he's talking about, so I'm
willing to dismiss this as a careless error.
Fortunately for you, there is some good nose: while I'm not completely certain, I believe I see a
small nose remnant on the face in Figure 2, just inferior to the eyes.
Confirmation of this spotting may require examination of an SEM, as the paper does with its
inclusion of Figure 4 (for which I applaud your diligence).

Overall, nice work, and although I was disheartened to discover your error, I was quite relieved
that your claims might still be true, if for the wrong reasons.
With that, I'll conclude and signal my approval.

\end{document}
