\documentclass[12pt]{article}
\usepackage{graphicx}
\usepackage{multirow}
\usepackage{amsmath}
\usepackage{fullpage}
\usepackage{times}
\usepackage{ulem}
\setlength\parindent{0pt}
\setlength\parskip{12pt}

\begin{document}

{\sffamily
\begin{tabular}{ll}
\multirow{3}{*}{\includegraphics[width=1in]{ach.png}}\\
& \Large{\em CONFIDENTIAL COMMITTEE MATERIALS} \\
&\\
& \textbf{\Huge{SIGBOVIK 2019 Paper Review}} \\
&\\
& \LARGE{Paper 23: Need more RAM?} \\[0.25em]
& \LARGE{Just invent time travel!} \\
&\\
\hline
\end{tabular}}
\vspace{2em}
\thispagestyle{empty}

{\large\bf
  \begin{tabular}{l}
    ``Anonymous'' Reviewers\\
Rating: Unjustifiably strong reject \\
Confidence: Expert (it's always the expert)\\
\end{tabular}}
\vspace{1em}

While this paper makes promising strides in the area of time-traveling
complexity theory, we simply cannot excuse the fact that the authors
don't include a citation to the work of Chapman, Katz and Muller [1] (no
relation to the reviewers). While the present submission differs from
the prior work by being in a different field of computer science, making
better pop-culture references and arguably involving actual scientific
merit, the idea of writing a SIGBOVIK paper about time travelling machines
is clearly
not novel and therefore it is our entirely objective opinion that the
submission should be rejected, or at least published alongside this
review so everyone can be reminded of how insightful \sout{we} the prior
authors are.

[1] Peter Chapman, Deby Katz and Stefan Muller. A Proposal for
Overhead-Free Dependency Management with Temporally Distributed
Virtualization. SIGBOVIK 2013.
\end{document}
