%%%%%%%%%%%%%%%%%%%%%%%%%%%%%%%%%%%%%%%%%%%%%%%%%
%%%
%%% SIGBOVIK 2020 TRIPLE BLIND REVIEW TEMPLATE
%%%
%%% Instructions:
%%%
%%% 1. Edit the author, rating, and confidence
%%%    fields.
%%% 2. Enter your review after the \maketitle
%%%    command
%%%
%%%%%%%%%%%%%%%%%%%%%%%%%%%%%%%%%%%%%%%%%%%%%%%%%
\documentclass[12pt]{sigbovik-review}

%%%% Edit the following three lines.
%% To ensure the integrity of the triple-blind review
%% process, the contents of the \author field should not
%% reveal your identity.
\author{Dos Sleddins, Time Detective}
\rating{Inevitable Accept}
\confidence{3.999991 - $\displaystyle\int_{2020}^{2071} \tau(y) \text{d}y$ \\
(where $\tau$ is Longwood's temporal inference decay function; see SIGBOVIK 2059.)}

%%%% The Proceedings Chair will fill in the following
%%%% two fields.
\papernum{NN}
\papertitle{UNKNOWN PAPER TITLE}

\begin{document}

\maketitle

Hey there, Harry. Long time no review---or at least, from your reference frame, anyway.
I've been hopping the chrono-eddies left and right, strange and charm
(in far excess of the BLCSC's recommended subjective-yearly intake, I might mention),
in search of the beating heart of SIGBOVIK's research causality web.
It's this paper.

I made sure Reviewer Two woke up in a good mood that one dreary March morning in 2010.
I snuck subtle citation formatting bugs upstream into 2033's new typesetting software.
All to nudge your legacy in the right direction
for Ringard's Paradox to finally be solved in 2071.
2020's triple-blind causality protections were the hardest to crack of all,
and for that I thank the PC; I truly do---but for reasons that would undo my own existence were I to utter them here.

And now that I'm now,
it almost feels like a formality to write a review for the Nexus itself.
Nevertheless:

Program committee, you must accept this paper.
No; it is simply inevitable that you {\it will} accept it.
Although it may seem to be of totally unrelated subject matter,
this paper lays the groundwork for
Highest-Order Logic
{\it and} for
the Lambda Timecube.
The inspiration it draws from $\beta$-Reduction Hero
is both unprecedented and unsuccedented,
although these authors were too humble to deign to spend even a footnote on it.
Its multidisciplinary take on the Call for Papers
will have opened fifteen new realms of study for Bovicians worldwide
in the years to come.

It could do with a few more explanatory pictures.

First time y'all been fully decorporealized this year, huh?
Well, ya pulled it off fantastic despite the circumstances.
Have hope---it gets better. Nice page numbers by the way.


\end{document}
