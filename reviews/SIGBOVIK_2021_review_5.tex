%%%%%%%%%%%%%%%%%%%%%%%%%%%%%%%%%%%%%%%%%%%%%%%%%
%%%
%%% SIGBOVIK 2020 TRIPLE BLIND REVIEW TEMPLATE
%%%
%%% Instructions:
%%%
%%% 1. Edit the author, rating, and confidence
%%%    fields.
%%% 2. Enter your review after the \maketitle
%%%    command
%%%
%%%%%%%%%%%%%%%%%%%%%%%%%%%%%%%%%%%%%%%%%%%%%%%%%
\documentclass[12pt]{sigbovik-review}
\usepackage{zxjatype}
\usepackage[ipa]{zxjafont}

%%%% Edit the following three lines.
%% To ensure the integrity of the triple-blind review
%% process, the contents of the \author field should not
%% reveal your identity.
\author{寿限無寿限無五劫の擦り切れ海砂利水魚の水行末雲来松風 \\
来末食う寝る処に住む処藪ら柑子の藪柑子パイポパイポパイポのシュ \\
ーリンガンシューリンガンのグーリンダイグーリンダイのポンポコピ \\
ーのポンポコナの長久命の長助}
\rating{Wow (somewhere between "ouch" and "boing")}
\confidence{dude trust me}

%%%% The Proceedings Chair will fill in the following
%%%% two fields.
\papernum{NN}
\papertitle{UNKNOWN PAPER TITLE}

\begin{document}

\maketitle

%%% Replace the following text with your review.
So here's the thing. I spent, like, 25 minutes reading this paper, and then another 30 minutes googling all the long words that sounded important, and I've come to the conclusion that I probably shouldn't be reviewing academic papers. It seemed like a fun idea at first, like, "Oh, just read this paper and give your opinions on it, readers like to see other perspectives," but it's just kinda overwhelming? I mean, when I think about how EVERYONE who reads the proceedings of this big conference is gonna look at what I wrote and use it to inform their own opinions, it just feels like too much responsibility.  
  
Basically what I'm trying to say is that I have no idea what this paper says. So instead I want you to make your own judgement on how good this paper is. Sure, maybe *I* found the author's postulation that the seven layers of the OSI model are analogous to the seven chakras to be a bit difficult to follow, but you don't have to let that affect your perception of the paper. Just because *I* don't know what "Hyperparadigmatistical n'-state macrocontrollers" are used for doesn't detract from the obvious wealth of knowledge that the author of this paper has blessed upon our mortal realm.  
  
There's definitely a lot of complicated, super important-sounding things going on in this paper, but I really just don't think I'm qualified to give any kind of commentary on it. Still, I wish all the best to anyone who can make sense of it and hope it goes on to revolutionize the field of... whatever field it belongs to.

\end{document}
