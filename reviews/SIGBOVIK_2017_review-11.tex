\documentclass[12pt]{article}
\usepackage{graphicx}
\usepackage{multirow}
\usepackage{amsmath}
\usepackage{fullpage}
\usepackage{times}
\setlength\parindent{0pt}
\setlength\parskip{12pt}

\begin{document}

{\sffamily
\begin{tabular}{ll}
\multirow{3}{*}{\includegraphics[width=1in]{ach.png}}\\
& \Large{\em CONFIDENTIAL COMMITTEE MATERIALS} \\
&\\
& \textbf{\Huge{SIGBOVIK 2017 Paper Review}} \\
&\\
& \LARGE{Paper 11: The Zero-Color Theorem}\\
& \LARGE{An Optimal Poster Design Algorithm} \\
&\\
\hline
\end{tabular}}
\vspace{2em}
\thispagestyle{empty}

{\large\bf
\begin{tabular}{l}
Stefan Muller, Paper Collector\\
Rating: 0/0\\
Confidence: NaN\\
\end{tabular}}
\vspace{1em}

This paper introduces and proposes a proof of the zero-color theorem.
I believe that the theorem, if true, would be an important advance in
simplifying the poster design process. Unfortunately, I am not convinced
by the result, which relies on the assumption that posters spend most of their
life cycle in a poster tube. This, in turn, presuposses that graduate students
can afford poster tubes, and do not simply attend poster sessions at their
own institutions by rolling the poster up in a cylinder and shimmying a
rubber band around it, the same state in which the poster will then find
itself (either propped up in a corner or placed on a bookshelf).
Keeping in mind this academic impoverishment factor,
the theorem must consider two additional cases. If the poster is rolled with
the printed side inward, the zero-color theorem may still hold. However, if
the printed side is outward, some of the poster will be permanently displayed
as it gathers dust. This means that the most important material
on the poster must be squeezed into a space the width of the poster and the
length $r\theta$, where $r$ is the radius of the rolled-up poster and
$\theta$ is the angle of the arc of poster which is visible. Within this strip,
more than zero colors must be used. The paper should explore the optimal ink
colors for this strip (most likely dark colors so that they will be visible
once the poster has begun to fade).

\end{document}
