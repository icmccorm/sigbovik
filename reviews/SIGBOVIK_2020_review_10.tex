%%%%%%%%%%%%%%%%%%%%%%%%%%%%%%%%%%%%%%%%%%%%%%%%%
%%%
%%% SIGBOVIK 2020 TRIPLE BLIND REVIEW TEMPLATE
%%%
%%% Instructions:
%%%
%%% 1. Edit the author, rating, and confidence
%%%    fields.
%%% 2. Enter your review after the \maketitle
%%%    command
%%%
%%%%%%%%%%%%%%%%%%%%%%%%%%%%%%%%%%%%%%%%%%%%%%%%%
\documentclass[12pt]{sigbovik-review}

%%%% Edit the following three lines.
%% To ensure the integrity of the triple-blind review
%% process, the contents of the \author field should not
%% reveal your identity.
\author{Definitely an expert and not a first-year grad student subreviewing}
\rating{Strongest of Rejects}
\confidence{Expert}

%%%% The Proceedings Chair will fill in the following
%%%% two fields.
\papernum{NN}
\papertitle{UNKNOWN PAPER TITLE}

\begin{document}

\maketitle

%%% Replace the following text with your review.
I really wanted to like this paper, because the topic is interesting and
because it is bad form to start reviewing a paper with the intention of
hating it. Unfortunately, I got confused at a number of points while
reading and just feel that the paper is too obscure for the SIGBOVIK
audience. Here are some examples drawn from the first page, which is
the only page that I can be sure the paper has.

\begin{enumerate}
\item I got completely lost on the second paragraph of p.1 and am sure that
the average reader will as well. It's definitely not just me.
\item Your use of the term ``it'' is confusing. Am I supposed to know what
``it'' is supposed to refer to?
\item SIGBOVIK readers come from a wide variety of backgrounds and may not
be familiar with what a ``computer'' is, though of course I am.
\item The definition of ``algorithm'', as I'm familiar with it, is
``a set of rules for solving a problem in a finite number of steps,
as for finding the greatest common divisor'' [Dictionary.com].
I don't believe you're using this term correctly, as your paper does not
mention greatest common divisors. It may not even mention algorithms, which
would be an even more glaring flaw in the paper.
\end{enumerate}


\end{document}
