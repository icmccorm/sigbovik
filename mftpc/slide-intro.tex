Your oscillators quiver in simulated excitement in anticipation of
the robot dance party in honor of $2^6$th birthdays---%
in particular, that of Harry Qbit Bovik---%
which you now approach.
With the doors to the SIGBOVIK 2018 dance party looming ahead,
you prefetch the 31 specimens of top-notch research,
readying yourself to discuss, enjoy, and perhaps even do follow-up work on
the brilliant research contained therein.
After scanning them in reverse-alphabetical order by first author name,
you are fully prepared.
You push open the daunting dance party doors...
and realize there's been a \texttt{MAX\_UINT64\_T}-sized mistake.

This isn't a robot dance party.
It's a human dance party!

Audio thumps in the artificially limited 20--20000~Hz range
as humans dance, converse,
and sip on dangerously conductive beverages.
You make your way through the crowd, attempting to appear human.
It's a dangerous world for robots:
any of these humans might be a Serious Researcher
who wants to reprogram you with Serious Research Code!

A new song starts---``This is something new, the Casper slide part two''---%
causing the dancing humans to cheer and form an approximate grid,
as if awaiting an order.
Having been caught in the crowd, you too must dance,
so you take your place in the grid.
You have no dancing programs compiled---not even \texttt{the\_robot.exe}---%
but the orderly grid-like formation gives you hope:
this may be one of the rare human songs
whose instructions are broadcast at runtime.
Hearing ``everybody clap your hands'' confirms this,
and you repeatedly bang together the ends of
your two general-purpose manipulators,
synchronizing with the music and the crowd of humans.

The instructions,
which are unfortunately given in a human natural-language ISA, continue.
You interpret ``to the left'' as ``move to the left'',
given that dancing often involves movement.
As you spin your wheels to locomote leftwards,
you are relieved to find the crowd of humans doing the same,
albeit sans wheels.
The next instruction is announced: ``take it back now, y'all''.

Disaster!
The number-of-possible-meanings register of natural language coprocessor
overflows.
What is the correct implementation of the ``take it back'' instruction?

Finding yourself in an ambiguous situation,
you are forced to invoke
the seldom-used \texttt{choose\_dear\_reader()} system call.
Your simulated spirits sink as you realize that
this is the sort of evening that will likely require repeated invocations.

\begin{switch}
\item{UNDO}
  Clearly ``take it back'' means ``undo''.
  \goto{bathroom-intro}
\item{BACKWARDS}
  Clearly ``take it back'' means ``move to the back''.
  \goto{slide-dance}
\item{REVENGE}
  Clearly ``take it back'' means
  ``take back that which is rightfully yours''.
  \linett{KILL\_ALL\_HUMANS()}
  \goto{end-alone}
\end{switch}


%%% Local Variables:
%%% mode: latex
%%% TeX-master: "message-from-pc"
%%% End:
